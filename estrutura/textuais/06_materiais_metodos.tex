\chapter{MATERIAIS E MÉTODOS}
\label{Material_metodo}

\section{Materiais}
\hspace{0.5cm} O trabalho tem caráter teórico e computacional com aplicação de técnicas de otimização matemática de programação não-linear, implementado no \textit{software} $GAMS$ (\textit{General Algebric Model System}), planilha eletrônica, \textit{software} Geogebra, \textit{PyCharm}.

\section{Métodos}

\hspace{0.5cm} Para o determinar o equilíbrio de fase sólido-líquido, foram aplicados os modelos termodinâmicos, por meio de cálculos pode-se determinar coeficientes ligados diretamente com a Energia livre de Gibbs.  \cite{Rocha2009a,Costa2007}

Segundo ROCHA \citeyear{Rocha2009a}, o diagrama de fase sólido-líquido para misturas binárias pode ser encontrado pela minimização da energia de Gibbs. O método para determinar o diagrama de fase sólido-líquido é capaz de incluir o número de componentes ($NC$) desejado, o valor de componentes para esse trabalho é de $NC=2$, com possibilidade futura de aplicações em misturas de mais componentes. A condição necessária para o mínimo, de componentes e produto, é dada pela convexidade do problema. Mas uma suposição implícita usada aqui no modelo é de que existe apenas uma fase líquida. Portanto para um modelo geral de atividade líquida deve ter uma análise cuidadosa, pois em modelos não-convexos, podem ser verificadas existências de outras possibilidades de equilíbrio, do tipo líquido-líquido-sólido, por exemplo. 

